%\documentclass[a4paper,twocolumn]{article} % Document type

\documentclass[a4paper,12pt,oneside,onecolumn]{article} % Document type

\usepackage[left=1.0in, right=1.0in, top=1.0in, bottom=1.0in]{geometry}

\ifx\pdfoutput\undefined
    %Use old Latex if PDFLatex does not work
   \usepackage[dvips]{graphicx}% To get graphics working
   \DeclareGraphicsExtensions{.eps} % Encapsulated PostScript
 \else
    %Use PDFLatex
   \usepackage[pdftex]{graphicx}% To get graphics working
   \DeclareGraphicsExtensions{.pdf,.jpg,.png,.mps} % Portable Document Format, Joint Photographic Experts Group, Portable Network Graphics, MetaPost
   \pdfcompresslevel=9
\fi

\usepackage{amsmath,amssymb}   % Contains mathematical symbols
\usepackage[ansinew]{inputenc} % Input encoding, identical to Windows 1252
\usepackage[english]{babel}    % Language
\usepackage[square,numbers]{natbib}     %Nice numbered citations
\bibliographystyle{plainnat}            %Sorted bibliography



\begin{document}               % Begins the document

\title{Paperwork on Degree Project: Machine Learning Based Fault Prediction for Real-time Scheduling on Shop-floor}
\author{
  First Name 1 Last Name 1 \\ Personal Number 1 \\ E-mail 1 
  \and 
  First Name 2 Last Name 2 \\ Personal Number 2 \\ E-mail 2
  }
%\date{2010-10-10}             % If you want to set the date yourself.

\maketitle                     % Generates the title




%%%%%%%%%%%%%%%%%%%%%%%%%%%%%%%%%%%%%%%%%%%%%%%%%%%%%%%%%%%%%%%%%%%%%%%%%%%%%%%%%%%
% Instructions regarding the report
%%%%%%%%%%%%%%%%%%%%%%%%%%%%%%%%%%%%%%%%%%%%%%%%%%%%%%%%%%%%%%%%%%%%%%%%%%%%%%%%%%%

\section*{Hopfield Network}

\section*{Overview}
Hopefield network is a form of recurrent artificial neural network described by Little in 1974 and popularized by John Hopfield in 1982. Despite its recurrent structure, a main feature of Hopfield network is the use of binary threshold units, meaning the units only take on two different values for their states and the value is determined by whether or not the units' input exceeds their threshold. Since Hopefield network has the ability of association, in machine learning meaning the ability to retrieve distorted patterns, it provides a way to understand human memory as well. \\
Many researches have been conducted on Hopfield network in recent years. For example Dayal Pyari Srivastava et al. used Hopefield network to model the microtubules in the brain(International Journal of General Systems)[1]. Yu.A.Basistov and Yu,G.Yanovskii performed comparison between Bayes, correlation algorithm and Hopfield network in the field of image recognition finding Bayes and Hopfield as equals, outperforming correlation algorithm in general(Pattern Recognition and Image Analysis)[2]. Furthermore, Hopfield network was also used in biochemistry by Quan Zou et al. for RNA secondary structure prediction(Computers in Biology and Medicine)[3]. Jiakai Li and Gursel Serpen equipped wireless sensor network with computational intelligence and adaptation capability using Hopfield network model(Applied Intelligence)[4].


\section*{Algorithm Details}
Overall, medium computation time, small amount of training data works. Capable of learning complex patterns. Limited storage capacity: 138 vectors for 1000 nodes.
Step 1:\\
...\\
Step n:
equations

%%%%%%%%%%%%%%%%%%%%%%%%%%%%%%%%%%%%%%%%%%%%%%%%%%%%%%%%%%%%%%%%%%%%%%%%%%%%%%%%%%%
% The bibliography
%%%%%%%%%%%%%%%%%%%%%%%%%%%%%%%%%%%%%%%%%%%%%%%%%%%%%%%%%%%%%%%%%%%%%%%%%%%%%%%%%%%
%\bibliography{Bibliography_template} %Read the bibliography from a separate file

\begin{thebibliography}{99}
\bibitem[Khalil(2002)]{Khalil:2002:Nonlinear-systems:vh}
Hassan~K Khalil.
\newblock \emph{Nonlinear systems}.
\newblock Prentice Hall, Upper Saddle river, 3. edition, 2002.
\newblock ISBN 0-13-067389-7.

\bibitem[Oetiker et~al.(2008)Oetiker, Partl, Hyna, and
  Schlegl]{Oetiker:2008:TheNotSoShortIntroductiontoLaTeXe}
Tobias Oetiker, Hubert Partl, Irene Hyna, and Elisabeth Schlegl.
\newblock \emph{The Not So Short Introduction to \LaTeXe}.
\newblock Oetiker, OETIKER+PARTNER AG, Aarweg 15, 4600 Olten, Switzerland,
  2008.
\newblock http://www.ctan.org/info/lshort/.

\bibitem[Sastry(1999)]{Sastry:1999:Nonlinear-systems:-analysis-stability-and-c%
ontrol:xr}
Shankar Sastry.
\newblock \emph{Nonlinear systems: analysis, stability, and control},
  volume~10.
\newblock Springer, New York, N.Y., 1999.
\newblock ISBN 0-387-98513-1.
\end{thebibliography}


\end{document}      % End of the document
