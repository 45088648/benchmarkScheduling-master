%\documentclass[a4paper,twocolumn]{article} % Document type

\documentclass[a4paper,12pt,oneside,onecolumn]{article} % Document type

\usepackage[left=1.0in, right=1.0in, top=1.0in, bottom=1.0in]{geometry}

\ifx\pdfoutput\undefined
    %Use old Latex if PDFLatex does not work
   \usepackage[dvips]{graphicx}% To get graphics working
   \DeclareGraphicsExtensions{.eps} % Encapsulated PostScript
 \else
    %Use PDFLatex
   \usepackage[pdftex]{graphicx}% To get graphics working
   \DeclareGraphicsExtensions{.pdf,.jpg,.png,.mps} % Portable Document Format, Joint Photographic Experts Group, Portable Network Graphics, MetaPost
   \pdfcompresslevel=9
\fi

\usepackage{amsmath,amssymb}   % Contains mathematical symbols
\usepackage[ansinew]{inputenc} % Input encoding, identical to Windows 1252
\usepackage[english]{babel}    % Language
\usepackage[square,numbers]{natbib}     %Nice numbered citations
\bibliographystyle{plainnat}            %Sorted bibliography



\begin{document}               % Begins the document

\title{Paperwork on Degree Project: Machine Learning Based Fault Prediction for Real-time Scheduling on Shop-floor}
\author{
  First Name 1 Last Name 1 \\ Personal Number 1 \\ E-mail 1 
  \and 
  First Name 2 Last Name 2 \\ Personal Number 2 \\ E-mail 2
  }
%\date{2010-10-10}             % If you want to set the date yourself.

\maketitle                     % Generates the title




%%%%%%%%%%%%%%%%%%%%%%%%%%%%%%%%%%%%%%%%%%%%%%%%%%%%%%%%%%%%%%%%%%%%%%%%%%%%%%%%%%%
% Instructions regarding the report
%%%%%%%%%%%%%%%%%%%%%%%%%%%%%%%%%%%%%%%%%%%%%%%%%%%%%%%%%%%%%%%%%%%%%%%%%%%%%%%%%%%

\section*{Radial Basis Function Network}
Radio basis function network(RBF network), first formulated in a 1988 paper by Broomhead and Lowe, is a class of artificial neural networks that uses radial basis functions as activation functions The output of the network is a linear combination of radial basis functions of the inputs and neuron parameters. Till now, RBF networks are widely used in  function approximation, time series prediction, classification, and system control.
\\
Barnali Dey et al. introduced computational intelligence to spectrum sensing in Cognitive Radio by using RBF network as a model and found their model better than conventional ones(Intelligent Automation and Soft Computing)[1]. Hitoshi Nishikawa and Seiichi Ozawa proposed a novel type of RBF network for multitask pattern recognition(Neural Processing Letters)[2]. H. Z. Dai et al. developed an improved RBF network for structural reliability analysis(Journal of Mechanical Science and Technology 25 (9) (2011) 2151~2159 )[3]. Mohammad Reza Sabour and Saman Moftakhari Anasori Movahed applied RBF neural network to predict soil sorption partition coefficient(Chemosphere)[4].

\section*{Overview}



\section*{Algorithm Details}
Overall, short computation time, medium model complexity. Capable of learning complex patterns. Good generalization ability. Similar to multilayer ANNs.\\
Step 1:\\
...\\
Step n:
equations

%%%%%%%%%%%%%%%%%%%%%%%%%%%%%%%%%%%%%%%%%%%%%%%%%%%%%%%%%%%%%%%%%%%%%%%%%%%%%%%%%%%
% The bibliography
%%%%%%%%%%%%%%%%%%%%%%%%%%%%%%%%%%%%%%%%%%%%%%%%%%%%%%%%%%%%%%%%%%%%%%%%%%%%%%%%%%%
%\bibliography{Bibliography_template} %Read the bibliography from a separate file

\begin{thebibliography}{99}
\bibitem[Khalil(2002)]{Khalil:2002:Nonlinear-systems:vh}
Hassan~K Khalil.
\newblock \emph{Nonlinear systems}.
\newblock Prentice Hall, Upper Saddle river, 3. edition, 2002.
\newblock ISBN 0-13-067389-7.

\bibitem[Oetiker et~al.(2008)Oetiker, Partl, Hyna, and
  Schlegl]{Oetiker:2008:TheNotSoShortIntroductiontoLaTeXe}
Tobias Oetiker, Hubert Partl, Irene Hyna, and Elisabeth Schlegl.
\newblock \emph{The Not So Short Introduction to \LaTeXe}.
\newblock Oetiker, OETIKER+PARTNER AG, Aarweg 15, 4600 Olten, Switzerland,
  2008.
\newblock http://www.ctan.org/info/lshort/.

\bibitem[Sastry(1999)]{Sastry:1999:Nonlinear-systems:-analysis-stability-and-c%
ontrol:xr}
Shankar Sastry.
\newblock \emph{Nonlinear systems: analysis, stability, and control},
  volume~10.
\newblock Springer, New York, N.Y., 1999.
\newblock ISBN 0-387-98513-1.
\end{thebibliography}


\end{document}      % End of the document
